\section*{Problem 1}
	\begin{proof}
		First, note that the \textit{Theorem}: [Composition of bijections is a bijection]. Let $N = \{1, 2, \cdots, n\}$. Since $f: N \to N$ is a bijection, $f^k: N \to N$ is also a bijection. Let $X = \{f\mid f: N \to N, f\mbox{ is a bijection}\}$. Then $f^k \in X$ for $^\forall k \in \mathbb{Z}^+$. Note that by the definition of $X$, the number of elements of $X$ is the same as the number of bijections from $N$ to $N$, so $|X| = n!$. Although $|X|$ is finite, there are countably many $f^k$ for $k \in \mathbb{Z}^+$. Therefore, by the \textit{Pigeonhole Principle}, $^\exists i, j \in \mathbb{Z}^+$ such that $f^i = f^j$ even if $i \neq j$.\\
	\end{proof}