\section*{Problem 5}
	\begin{proof} [Solution]
		\mbox{}\\질문: 현대의 채용 프로세스는 컴퓨터를 통해 인적성검사를 실시한다. 정해진 답이 있는 것처럼, 문항에 따라 평가 지표가 만들어지고 관리된다. 이것이 사람의 평가보다 공정하다고 할 수 있을까? 평가 알고리즘 또한 사람이 만들었기에 공정하지 않을 수 있고, 오류나 차별이 발생해도 불투명한 과정으로 인해 불이익을 당하는 사람이 있을 수 있다. 현대의 채용 프로세스가 공정하지 못하다면, 어떻게 해결할 수 있을까?\\
		
		답변: 현대의 채용 프로세스가 공정하지 못할 수 있는 이유에 대해 말씀드리겠습니다. 
		\begin{enumerate} [1.]
			\item \textbf{알고리즘의 편향}: 알고리즘은 그것을 개발한 사람들의 편견이나 선입견을 반영할 수 있습니다. 예를 들어, 특정 인구 그룹에 대한 선입견이나 차별이 반영되어 있을 수 있습니다.
			\item \textbf{데이터의 편향}: 알고리즘은 학습 데이터를 기반으로 작동합니다. 이 데이터가 특정 인구 그룹을 대표하지 않거나, 특정 그룹에 대한 편향이 있으면, 그 결과도 공정하지 않을 수 있습니다.
			\item \textbf{투명성의 부족}: 어떤 평가 기준으로 어떻게 평가되는지에 대한 투명성이 부족할 경우, 신청자들은 자신이 왜 불이익을 받게 되었는지 이해하기 어렵습니다.
		\end{enumerate}
		
		이러한 문제를 해결하기 위해 몇 가지 방안이 있습니다:
		\begin{enumerate} [1.]
			\item \textbf{다양한 의견 수렴}: 평가 알고리즘을 개발할 때 다양한 관점을 수렴시키고 다양한 인종, 성별, 백그라운드의 사람들을 참여시켜야 합니다.
			\item \textbf{데이터의 다양성 보장}: 학습 데이터를 수집할 때, 가능한 모든 인구 그룹을 대표할 수 있는 다양성을 확보해야 합니다.
			\item \textbf{투명성 확보}: 알고리즘이 어떻게 작동하는지, 어떤 평가 기준으로 평가되는지를 설명하는 투명한 문서를 작성하고, 지원자들에게 공개해야 합니다.
			\item \textbf{지속적인 감시와 개선}: 채용 프로세스와 알고리즘은 지속적으로 모니터링되어야 하며, 발견된 문제는 빠르게 수정되어야 합니다.
			\item \textbf{피드백 반영}: 지원자들로부터의 피드백을 수렴하고, 그를 토대로 프로세스를 개선해야 합니다.
		\end{enumerate}
		
		이러한 접근 방법들을 통해 공정한 채용 프로세스를 만들 수 있습니다. 또한, 인적성 검사 결과를 토대로 최종 판단을 내리기 전에 다양한 평가 방법을 사용하여 종합적으로 평가하는 것도 중요합니다.\\
		
		비교: ChatGPT는 상당히 비슷한 해결 방안을 제시했다. 이것이 보편적으로 할 수 있는 방안이기 때문일 것이다. 하지만 ChatGPT는 분석과 개선을 어떻게 해야하는 가에 대한 조금 더 구체적인 답변을 제시했다. 많은 사람의 의견을 반영하고, 학습 데이터에 대표성을 보장해야하며, 알고리즘에 대한 문서를 공개할 필요가 있다고 언급했다.\\
	\end{proof}