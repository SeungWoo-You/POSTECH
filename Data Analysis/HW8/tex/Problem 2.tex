\section*{Problem 2}
	\begin{proof}
		Define a set of $k$-simplex be $\Sigma_k$. First, consider $X$. We need $k=0,1,2$.
		\begin{align*}
			\Sigma_0 &= \{[x_0], [x_1], [x_2], [x_3], [x_4]\}\\
			\Sigma_1 &= \{[x_0,x_1], [x_1,x_2], [x_0,x_2], [x_2,x_3], [x_2,x_4], [x_3,x_4]\}\\
			\Sigma_2 &= \emptyset
		\end{align*}
		Since there is only one graph, $\beta_0 = 1$. For $\beta_1$, define the first chain module $C_1$. for $\sigma\in C_1$,
		\begin{align*}
			\begin{multlined}[t]
				\sigma = t_1[x_0, x_1] + t_2[x_0,x_2] + t_3[x_1,x_2]\\
				+ t_4[x_2,x_3] + t_5[x_2,x_4] + t_6[x_3,x_4] \quad\mbox{ for } t_i\in \{0, 1\}
			\end{multlined}
		\end{align*}
		By the boundary operator $\partial_1$, we get
		\begin{align*}
			\partial_1\sigma &= \begin{multlined}[t]
				t_1([x_0]+[x_1]) + t_2([x_0]+[x_2]) + t_3([x_1]+[x_2])\\
				+ t_4([x_2]+[x_3]) + t_5([x_2]+[x_4]) + t_6([x_3]+[x_4])
			\end{multlined}\\
			&= \begin{multlined}[t]
				(t_1+t_2)[x_0] + (t_1+t_3)[x_1] + (t_2+t_3+t_4+t_5)[x_2]\\
				+ (t_4+t_6)[x_3] + (t_5+t_6)[x_4]
			\end{multlined}
		\end{align*}
		Find solutions for $t_i$. This means we need to solve the below matrix equation:
		\begin{equation*}
			\begin{bmatrix} 
				1 & 1 & 0 & 0 & 0 & 0 \\
				1 & 0 & 1 & 0 & 0 & 0 \\
				0 & 1 & 1 & 1 & 1 & 0 \\
				0 & 0 & 0 & 1 & 0 & 1 \\
				0 & 0 & 0 & 0 & 1 & 1
			\end{bmatrix}
			\begin{bmatrix} 
				t_1 \\
				t_2 \\
				t_3 \\
				t_4 \\
				t_5 \\
				t_6
			\end{bmatrix} = 
			\begin{bmatrix} 
				0 \\
				0 \\
				0 \\
				0 \\
				0
			\end{bmatrix}
		\end{equation*}
		From this, we get
		\begin{equation*}
			\begin{multlined}[t]
				\sigma = t_1([x_0, x_1] + [x_0,x_2] + [x_1,x_2])\\
				+ t_4([x_2,x_3] + [x_2,x_4] + [x_3,x_4])
			\end{multlined}
		\end{equation*}
		This means there are 2 cycles: $x_0-x_1-x_2$ and $x_2-x_3-x_4$. Since $\Sigma_2=\emptyset$, we have no plane. Therefore, $\beta_1 = 2 - 0 = 2$.\\
		Similarly, for $Y$, $\beta_0 = 1$. But $Y$ has $\Sigma_2=\{[x_2,x_3,x_4]\}$. This means there is a plane $x_2-x_3-x_4$. Therefore, $\beta_1 = 2 - 1 = 1$.\\
	\end{proof}