\section*{Problem 3}
	$\ulcorner$대량살상수학무기 Chapter 6. 디지털 골상학: 당신은 우리가 원하는 직원이 아닙니다$\lrcorner$\\
	요약: 6장은 많은 기업이 채용 과정에서 활용하는 인적성검사를 설명한다. 최근 인적성검사는 컴퓨터를 통해 실시하고 평가한다. 이는 정해진 답이 있는 것처럼 문항에 따라 평가 지표가 만들어진다. 물론 사람의 평가보다는 공정할 수 있다. 하지만 평가 알고리즘 또한 사람이 만든 것이기에, 편견이나 오류가 발생할 수 있다. 게다가 불합격한 사람은 정확한 이유를 알 수 없기에 차별이 발생해도 이를 눈치챌 수 없다. 인적성검사 과정이 매우 불투명하고 수정 불가능한 모델에 기반하기 때문이다. 그럼에도 이러한 절차가 여전히 사용되는 이유는 선발 과정의 편리함 덕분이다. 기업 입장에서는 인재를 선발하는 과정에 소요되는 시간과 비용을 최소화하는 것이 중요하다. 기업에 입사를 희망하는 지원자는 여전히 많으며, 컴퓨터 프로그램을 이용한다면 몇 명의 인재를 놓치더라도 그들을 편리하게 선별할 수 있다. 사람이 하나하나 선별한다고 하더라도 모든 인재를 선별할 수 있을 지는 미지수이며, 투자 대비 결과가 그렇게 좋지 않을 수 있다. 따라서 이러한 시스템은 여전히 유지되고 있으며, 프로그램에 대한 정보와 이를 쟁취할 부를 가진 사람들이 경쟁에서 우위를 차지할 가능성이 높아진다.\\
	해결 방안: 기업에서 평가 기준을 투명하게 공개하고, 알고리즘을 개선하고 검증하는 과정이 지속해서 이루어져야 할 것이다. 또한 컴퓨터 평가 결과를 주로 삼지 말고, 보조 용도로 사용하면 좋을 것이다. 애매한 경우에 대해 다시 한 번 사람이 보고 판단해 불합리한 탈락이 발생하지 않도록 해야 한다. 또한 이러한 프로그램을 기업에 부적한 사람을 배제하기 위한 용도가 아니라 필요한 자원을 제공하는 용도로 사용한다면 더 훌륭한 인재를 발굴할 수 있을 것이다.