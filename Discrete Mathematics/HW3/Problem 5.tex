\section*{Problem 5}
	\begin{enumerate} [(a)]
		\item 
		\begin{proof}
			Proof by induction. Let $P(j)$ be the given statement. Consider $j = 1$. Then $\sum\limits_{k = 1}^1 f_k^2 = f_1^2 = 1^2 = f_1f_2$ is clearly true.\\
			Suppose $P(j)$ is true for $j = n$. Consider $j = n + 1$. Note that the recursive relation of \textit{Fibonacci Sequence}: $f_n = f_{n - 1} + f_{n - 2}$.
			\begin{align*}
				\sum\limits_{k = 1}^{n + 1} f_k^2 &= \sum\limits_{k = 1}^{n} f_k^2 + f_{n + 1}^2\\
				&= f_nf_{n + 1} + f_{n + 1}^2\\
				&= f_{n + 1}(f_n + f_{n + 1})\\
				&= f_{n + 1}f_{n + 2}
			\end{align*}
			Therefore, $P(j)$ is true for $^\forall j\in\mathbb{N}$.\\
		\end{proof}
		\item 
		\begin{proof}
			Proof by induction. Note that the following:
			\begin{align*}
				&f_n = \frac{f_{n - 1} + \sqrt{5f_{n - 1}^2 + 4(-1)^{n + 1}}}{2}\\
				\Leftrightarrow&\ (2f_n - f_{n - 1})^2 = 5f_{n - 1}^2 + 4(-1)^{n + 1}\\
				\Leftrightarrow&\ 4f_n^2 - 4f_nf_{n - 1} + f_{n - 1}^2 = 5f_{n - 1}^2 + 4(-1)^{n + 1}\\
				\Leftrightarrow&\ 4f_n^2 - 4f_nf_{n - 1} = 4f_{n - 1}^2 + 4(-1)^{n + 1}\\
				\Leftrightarrow&\ f_n^2 - f_nf_{n - 1} = f_{n - 1}^2 + (-1)^{n + 1}\\
				\Leftrightarrow&\ f_n^2 = f_nf_{n - 1} + f_{n - 1}^2 + (-1)^{n + 1}\\
				\Leftrightarrow&\ f_n^2 = f_{n - 1}(f_n + f_{n - 1}) + (-1)^{n + 1}\\
				\Leftrightarrow&\ f_n^2 = f_{n - 1}f_{n + 1} + (-1)^{n + 1}
			\end{align*}
			Using this, let the last equation be $P(j = n)$. Consider $j = 2$. Then $f_2^2 = 1^2 = 0 = 1 \cdot 2 + (-1)^3 = f_{2 - 1}f_{2 + 1} + (-1)^{2 + 1}$ is true.\\
			Suppose $P(j = n)$ is true. Consider $j = n + 1$.
			\begin{align*}
				f_{n + 1}^2 &= f_{n + 1}(f_n + f_{n - 1})\\
				&= f_{n + 1}f_n + f_{n + 1}f_{n - 1}\\
				&= f_{n + 1}f_n + \left(f_n^2 - (-1)^{n + 1}\right)\\
				&= f_n(f_{n + 1} + f_{n}) + (-1)^{n + 2}\\
				&= f_nf_{n + 2} + (-1)^{n + 2}
			\end{align*}
			Therefore, $P(j)$ is true for $j \geq 2$.\\
		\end{proof}
		\item 
		\begin{proof}
			Proof by strong induction. Let $P(j)$ be the given statement. Consider $j = 6$. Then $f_6 = 8 > \left(\frac{3}{2}\right)^{6 - 1} \simeq 7.59$ is clearly true.\\
			Suppose $P(j)$ is true for $j = 1, 2, \cdots, n$. Consider $j = n + 1$.
			\begin{align*}
				f_{n + 1} &= f_n + f_{n - 1}\\
				&> \left(\frac{3}{2}\right)^{n - 1} + \left(\frac{3}{2}\right)^{n - 2}\\
				&= \left(\frac{3}{2}\right)^{n - 2}\left(\frac{3}{2} + 1\right)
				= \frac{5}{2}\left(\frac{3}{2}\right)^{n - 2}\\
				&= \frac{10}{4}\left(\frac{3}{2}\right)^{n - 2} > \frac{9}{4}\left(\frac{3}{2}\right)^{n - 2}\\
				&= \left(\frac{3}{2}\right)^2\left(\frac{3}{2}\right)^{n - 2} = \left(\frac{3}{2}\right)^n
			\end{align*}
			Therefore, $P(j)$ is true for $j \geq 6$.\\
		\end{proof}
		\item 
		\begin{proof}
			First, claim that $\gcd(a, b) = \gcd(a + b, b).$ Let $\gcd(a, b) = d$. Then $a = pd$, $b = qd$, $\gcd(p, q) = 1$ for some $p, q \in \mathbb{N}$. This gives $a + b = (p + q)d$. If $\gcd(p + q, q) = 1$, then $\gcd(a + b, b) = d$. Assume, if not, $\gcd(p + q, q) = c \neq 1$. Then $p + q = kc$, $q = tc$, $\gcd(k, t) = 1$ for some $k, t \in \mathbb{N}$. This gives $p = (k - t)c$, so $\gcd(p, q) \geq c$. But it contradicts to $\gcd(p, q) = 1$. Therefore, $\gcd(a + b, b) = d$. Now, $\gcd(a, b) = d \implies \gcd(a + b, b)$ is proved. We can proof the reveresed direction($\gcd(a + b, b) = d \implies \gcd(a, b)$) similarly. Therefore, the claim is true.\\
			Proof by induction. Let $P(j)$ be the given statement. Consider $j = 1$. Then $\gcd(f_1, f_2) = 1$ is clearly true.\\
			Suppose $P(j = n)$ is true. Consider $j = n + 1$. Then by the claim, $\gcd(f_{n + 1}, f_{n + 2}) = \gcd(f_{n + 1}, f_{n} + f_{n + 1}) = \gcd(f_{n}, f_{n + 1}) = 1$. Therefore, $P(j)$ is true for $^\forall j\in\mathbb{N}$.\\
		\end{proof}
	\end{enumerate}