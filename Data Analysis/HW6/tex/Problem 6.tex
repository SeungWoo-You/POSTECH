\section*{Problem 6}
	\begin{proof} [Solution]
		저자는 빅데이터가 지닌 문제를 어떻게 스몰데이터로 극복할 수 있는지 설명한다. 빅데이터는 상관관계 파악에 유리하지만, 빅데이터 자체로는 통찰력을 이끌어낼 수 없으며 분석 중심으로 이루어져 있다. 그렇기에 빅데이터만으로는 개개인의 성향이나 기호에 맞춘 분석이 어려울 수 있다. 하지만 스몰데이터를 이용하면 이를 효과적으로 파악할 수 있다. 따라서 스몰데이터와 빅데이터의 장점을 모두 활용해야 진정으로 상황에 맞게 적절한 분석이 가능하다는 입장이다.\\
		실제로 빅데이터 기반의 공부를 할 때, 빅데이터가 가진 장점에 대해서만 배웠던 기억이 있다. 우리는 기존의 감정적인 결정에서 벗어나, 빅데이터와 분석에 기반한 결정을 내릴 수 있어야 한다고 설명한다. 또한 빅데이터는 작은 변동을 무시하고 전체의 변화를 파악한다. 수 많은 데이터가 넘쳐나는 현대사회에는 빅데이터 시대가 불가피하다고 생각한다. 그래도 저자의 주장을 통해 그럼에도 스몰데이터만이 할 수 있는 일이 있다고 느꼈다. 빅데이터가 하지 못하는, 수 많은 데이터의 경향에 벗어나는 값을 세세하게 분석하는 등의 방법으로 스몰데이터를 활용할 수 있을 것이다.\\
	\end{proof}