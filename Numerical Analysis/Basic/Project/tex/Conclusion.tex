\section{Conclusion}
	This article analyzed the Black-Scholes model in analytic and numerical ways. Just Euler method is used for $t$ coordinate, so applying another ODE solving skills such as Runge-Kutta good for next project. From the Comparision section, it seems that spline method is more exact than midpoint formula. However, it should solve the $N+1$ system of equations for just one fixed time, so has exorbitant time complexity if $M$ is large. In addition, this method looks like very unstable. The stability and convergence conditions are not found in here, but in test, spline is more unstable than midpoint. Presumably, a huge $t$-step is required to slightly increase the $s$-step. Also, spline and interpolation does not guarantee that they converge to exact function. The global error does not decrease as $t\rightarrow T$. Hence, it seems that spline method is not suitable for large $N$ and $M$. However, It has been shown that appropriate polynomials can have good approximations. So further research needs to focus on finding proper polynomials such as orthogonal or Laguerre, and checking
	stable conditions of each method.\\
\section*{Reference}
	\begin{enumerate}
		\item Fisher, R. (2021). Comparing Numerical Solution Methods for the Cahn-Hilliard Equation.\\doi:10.1137/20S1357974.
		\item Jiawei He. (2021). A Study on Analytical and Numerical Solutions of Three
		Types of Black-Scholes Models. \textit{International Journal of Trade, Economics and Finance}(vol.12, 4). 109-114.\\doi:10.18178/ijtef.2021.12.4.703
		\item Nurul Anwar, M. and Sazzad Andallah, L. (2018). A Study on Numerical Solution of Black-Scholes Model. \textit{Journal of Mathematical Finance, 8}, 372-381. doi:10.4236/jmf.2018.82024.
		\item Richard L. Burden, J. Douglas Faires, and Annette M. Burden. (2015). \textit{Numerical Analysis}(10th ed.). Boston: Cengage Learning.
		\item Stein, Elias M. and Shakarchi, Rami. (2003). \textit{Fourier Analysis: an introduction}. NJ: Princeton University Press.
	\end{enumerate}