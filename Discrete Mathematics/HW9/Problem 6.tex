\section*{Problem 6}
	\begin{proof}
		Let $G=(V, E)$ be the graph. Let $T_1 = (V, E_1)$ and $T_2=(V,E_2)$ be two MSTs of $G$. If we show $T_1 = T_2$, then the proof is done. Note that consider $V$ and $E$ as sets.\\
		Proof by contradiction. Assume $T_1 \neq T_2$. This means $E_1 \neq E_2$. Since $\left|E_1\right| = \left|V\right| - 1 = \left|E_2\right|$(by the definition of \textit{tree}), $E_1 \neq E_2$ implies $E_1 - E_2 \neq \emptyset$ and $E_2 - E_1 \neq \emptyset$. WLOG, just consider the first-case: $E_1 - E_2 \neq \emptyset$. Let $e\in E_1 - E_2$. We can pick such edge $e$ because of non-empty. Since $e \not\in E_2$, the new subgraph $\Gamma_2 = (V, E_2\cup\{e\})$ has a cycle $C$. Since all weights in $G$ are distinct, there is a edge $e^\prime\in C$ which has the largest weight. Claim that $e^\prime\not\in$ any MST.
		\begin{itemize}
			\item [] \begin{proof} [Proof of the claim]
				Proof by contradiction. If $e^\prime \in T$ which is MST, then deleting $e^\prime$ will break $T$ into two subtrees. Since $e^\prime$ is from the cycle $C$, we can pick another edge $f$ which connects the two subetrees. By the definition of $e^\prime$, $e^\prime>f$. This gives $T$ is not MST because we can make a tree with smaller weight using $f$. Therefore, $e^\prime\not\in T$.
			\end{proof}
		\end{itemize}
		Note that the above claim is called \textit{cycle property}. Since $e^\prime\in C$ and $C\subseteq E_2\cup\{e\}$, $e^\prime\in E_2\cup\{e\}$. Here, we have two cases:
		\begin{itemize}
			\item If $e^\prime = e$, then since $e\in E_1 - E_2\subset E_1$, $e^\prime \in E_1$. But $T_1 = (V, E_1)$ is a MST, contradicts to the claim.
			\item If $e^\prime \in E_2$, but then $T_2 = (V, E_2)$ is a MST, contradicts to the claim.
		\end{itemize}
		Therefore, our assumption $T_1 \neq T_2$ is false.\\
	\end{proof}